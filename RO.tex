\documentclass[12pt, twoside, letterpaper]{article}
\usepackage[top=2cm,bottom=4cm,left=3cm,right=3cm,asymmetric]{geometry}% aggiungere ^twoside^
\usepackage{color}   %May be necessary if you want to color links
\usepackage{hyperref}
\hypersetup{
    colorlinks=true, %set true if you want colored links
    linktoc=all,     %set to all if you want both sections and subsections linked
    linkcolor=black,  %choose some color if you want links to stand out
}
\usepackage[utf8x]{inputenc}
\usepackage[table]{xcolor}
\usepackage[english]{babel}
\usepackage{amsmath, amsthm, amssymb, amsfonts}
\usepackage[breakable]{tcolorbox}
\newtcolorbox{dati}[1][]{colback=green!30!white, bottomtitle=1.5mm,breakable,#1}
\newtcolorbox{variabili}[1][]{colback=red!30!white, bottomtitle=1.5mm,breakable,#1}
\newtcolorbox{vincoli}[1][]{colback=orange!30!white, bottomtitle=1.5mm,breakable,#1}
\newtcolorbox{obiettivo}[1][]{colback=yellow!30!white, bottomtitle=1.5mm,breakable,#1}
\newcommand{\problema}[5]{
	#1
	\begin{dati}
		\paragraph{Dati} #2
	\end{dati}
	\begin{variabili}
		\paragraph{Variabili} #3
	\end{variabili}
	\begin{obiettivo}
		\paragraph{Obiettivo} #4
	\end{obiettivo}
	\begin{vincoli}
		\paragraph{Vincoli} #5
	\end{vincoli}
}
\newcommand{\problemai}[5]{
	#1
	\begin{dati}
		\paragraph{Dati} 
			\begin{itemize}
				#2
			\end{itemize}
	\end{dati}
	\begin{variabili}
		\paragraph{Variabili} 
			\begin{itemize}
				#3
			\end{itemize}
	\end{variabili}
	\begin{obiettivo}
		\paragraph{Obiettivo} 
			\begin{itemize}
				#4
			\end{itemize}
	\end{obiettivo}
	\begin{vincoli}
		\paragraph{Vincoli}
			\begin{itemize}
				#5
			\end{itemize}
	\end{vincoli}
}

\title{Ricerca operativa}
\author{Mario Petruccelli \cr Università degli studi di Milano}
\date{A.A. 2018/2019}

\addto\captionsenglish{% Replace "english" with the language you use
  \renewcommand{\contentsname}%
    {Sommario}%
}

\begin{document}

	\begin{titlepage}
		\maketitle
		\newpage
		\tableofcontents
	\end{titlepage}


	\section{Introduzione}
	
		\textbf{Ricerca operativa:} disciplina che affronta la risoluzione di problemi decisionali complessi tramite 	modelli matematici e algoritmi.
	Si parte da un \textbf{sistema organizzato} e lo si formalizza in un \textbf{modello matematico} per poi risolverlo tramite \textbf{algoritmi}.
		\subsection{Tassonomia modelli}
		\begin{itemize}
			\item \textbf{Descrittivi} $\rightarrow$ Modelli che cercano di descrivere o simulare sistemi complessi \textit{(e.g. modellini, plastici, \dots)}
			\item \textbf{Predittivi} $\rightarrow$ Modelli che cercano di predire dei dati \textit{(e.g. andamento mercati finanziari, previsioni, \dots)}
			\item \textbf{Prescrittivi} $\rightarrow$ Modelli che trovano la soluzione ottimale ad un problema \textit{(sono quelli che studieremo in questo corso)}. 
			\\\\La descrizione del problema avverrà attraverso \textbf{vincoli}, \textbf{obiettivi}.
		\end{itemize}
		\paragraph{Esempio di problemi decisionali}
		\begin{itemize}
			\item Finanza (investimenti)
			\item Produzione (dimensionamento, organizzazione, \dots)
			\item Logistica (gestione scorte, quanta merce, \dots)
			\item Gestione (pianificazione, turnistica personale, \dots)
			\item Servizi (rotte, \dots)
		\end{itemize}
		\paragraph{NB}\textit{Lo stesso modello può servire per risolvere problemi diversi.}
		\\\\\textbf{Set covering }Problema per la gestione di un territorio. I problemi dei sismografi e dei ripetitori sono diversi ma si ragiona allo stesso modo.
		\subsection{Programmazione matematica} La programmazione matematica (intesa come \textit{pianificazione} delle azioni necessarie per individuare la soluzione ottima) è ciò che rappresenta il processo risolutivo nella ricerca operativa: 
			\begin{itemize}
				\item Analisi del problema e scrittura di un modello matematico.
				\item Definizione e applicazione di un metodo di soluzione.
			\end{itemize}
		In particolar modo, la programmazione matematica si occupa di ottimizzare una funzione di più variabili, spesso soggette a dei vincoli. A seconda del tipo di modello abbiamo:
		\begin{itemize}
			\item Programmazione lineare continua.
			\item Programmazione lineare intera.
			\item Programmazione booleana.
		\end{itemize}
	
	\section{Esempi di problemi}
	
		\subsection{Problema dello zaino} 
				\problema{
					Ci sono $n$ oggetti di valore $p_j$ e ingombro $w_j$ per $j=1 \dots n$ ed è data la capacità massima $b$ di un contenitore.
			\paragraph{Problema} Quali oggetti inserire nel contenitore senza superare capacità. 
			\paragraph{Obiettivo}Massimizzare il valore degli oggetti. Si tratta di un problema di \textbf{ottimizzazione} e va formalizzato in modello matematico. Ci sono 4 componenti fondamentali.}
				{I dati sono informazioni conosciute a priori, in questo caso sono:
				\begin{itemize}
					\item $p_j \rightarrow$ valore dell'oggetto $j$.
					\item $w_j \rightarrow$ ingombro dell'oggetto $j$.
					\item $b \rightarrow$ capacità massima del contenitore.
				\end{itemize}}
				{Le variabili sono elementi che rappresentano una decisione. 
				\begin{itemize}
					\item $x_j= \begin{cases} \text{1 se il j-esimo oggetto viene inserito} \\ \text{0 altrimenti}\end{cases}$
				\end{itemize}}
				{L'obiettivo è la funzione che rappresenta il risultato da ottenere.
				\begin{itemize}
					\item $max \sum_{j=1}^np_jx_j \rightarrow \textit{massimizzare il valore}$
				\end{itemize}				  }
				{I vincoli sono le limitazioni che abbiamo sui dati. 
				\begin{itemize}
					\item $\sum_{j=1}^n w_j x_j \leq b \rightarrow$ la somma degli ingombri degli oggetti presi non può superare la capacità del contenitore 
					 \item $x_j \in \{0,1\} \quad j=1 \dots n$
				\end{itemize}}
		
		\subsection{Problema di trasporto e localizzazione di impianti}
			\problema
			{Ci sono $n$ siti candidati ad ospitare unità produttive, ciascuno con capacità massima $a_i$ con $i=1 \dots n$. Vi sono $m$ magazzini, ognuno con una domanda da soddisfare $b_j$ con $j = 1 \dots m$. Indichiamo con $c_{ij}$ il costo di trasporto di una unità di prodotto dal sito $i$ al magazzino $j$. L'attivazione di una unità produttiva nel sito $i$ ha un costo fisso $f_i$. 
			\paragraph{Problema} Dove aprire le unità produttive e come trasportare il prodotto dalle unità produttive aperte ai magazini in modo da soddisfare la domanda.
			\paragraph{Obiettivo} Minimizzare i costi di apertura e trasporto.}
			{\begin{itemize}
				\item $a_i \rightarrow$ capacità di produzione del sito $i$
				\item $b_j \rightarrow$ domanda del magazzino $j$ 
				\item $c_{ij} \rightarrow$ costo del trasporto di un'unità dal sito $i$ al magazzino $j$. 
				\item $f_i \rightarrow$ costo di attivazione unità nel sito $i$.  
			\end{itemize}} 
			{\begin{itemize} 
				\item $y_i = \begin{cases} \text{1 se il sito $i$ ospita un'unità produttiva} \\ \text{0 altrimenti}\end{cases}$
				\item $x_{ij} =$ numero di unità trasportata dal sito $i$ al magazzino $j$.
			\end{itemize}}
			{\begin{itemize}
				\item $min \sum_{i=1}^n \sum_{j=1}^m c_{ij}x_{ij} + \sum_{i=1}^n f_i y_i \rightarrow$ minimizzare il costo di attivazione di un unità nei vari siti e il costo dei trasporti delle unità.
			\end{itemize}}
			{\begin{itemize}
				\item $\sum_{j=1}^m x_{ij} \leq a_i y_i \quad i = 1 \dots n \rightarrow$ le unità trasportate da un sito i non possono superare la capacità $a_i$ di quel sito $i$.
				\item $\sum_{i=1}^n x_{ij} \geq b_j \quad j=1 \dots m \rightarrow$ Le unità inviate ad un magazzino $j$ dai vari siti deve soddisfare la domanda di quel magazzino.
				\item $x_{ij} \geq 0 \quad i=1 \dots n \quad j = 1 \dots m$
				\item $y_i \in \{0,1\} \quad i=1 \dots n$
			\end{itemize}}
						
		\subsection{Problema assegnamento}					
			\problemai
			{Ci sono $n$ lavoratori e $n$ attività. Indichiamo con $t_{ij}$ il tempo impiegato dal lavoratore $i$ per svolgere l'attività $j$.
			\paragraph{Problema} Assegnare a ciascun lavoratore una sola attività, così che tutte le attività siano svolte.
			\paragraph{Obiettivo} Minimizzare il tempo richiesto a svolgere l'attività $j$.}
			{\item $t_{ij} \rightarrow$ tempo impiegato dal lavoratore $i$ per svolgere l'attività $j$.}
			{\item $x_{ij} = \begin{cases} \text{1 se il lavoratore $i$ svolge l'attività $j$} \\ \text{0 altrimenti}\end{cases}$ }
			{\item $min \sum_{i=1}^n\sum_{j=1}^n t_{ij} x_{ij} \rightarrow$ minimizzare il tempo speso per svolgere tutte le attività dei vari lavoratori.}
			{\item $\sum_{j=1}^n x_{ij} = 1 \quad \forall i \rightarrow$ a ogni lavoratore è associata una sola attività.
			\item $\sum_{i=1}^n x_{ij} = 1 \quad \forall j \rightarrow$ a ogni attività è associata nn solo lavoratore.
			\item $x_{ij} \in \{0,1\} \quad \forall i,j$}

		
		\subsection{Mix Produttivo}
			\problemai
				{Si hanno $m$ risorse produttive con disponibilità $b_i$. Si possono produrre $n$ prodotti diversi. Per produrre una unità di un prodotto \textit{j-esimo} si utilizzano $a_{ij}$ unità della risorsa \textit{i-esima}. Ciascun prodotto ha un profitto unitario $c_j$.}
				{\item $b_i \rightarrow$ disponibilità risorsa \textit{i-esima}.
				\item $a_{ij} \rightarrow$ unità della risorsa \textit{i-esima} usate per produrre un prodotto \textit{j-esimo}.
				\item $c_j \rightarrow$ profitto di un unità del prodotto $j$. }
				{\item $x_j =$ unità prodotte del prodotto \textit{j-esimo}.}
				{\item $max \sum_{j=1}^n c_j x_j \rightarrow$ massimizzare il profitto tra i vari prodotti.}
				{\item $\sum_{j=1}^n a_{ij} x_j \leq b_i \quad \forall i \rightarrow$ le risorse usate nella produzione non possono superare la disponibilità di ciascuna risorsa.
				\item $x_j \geq 0 \quad \forall j$}


			\subsubsection{Vernici}
				\problema
					{L'azienda produce due tipi di vernici, una vernice per interni (I) e una venrice per esterni (E), usando due materie prime indicate con A e B. La disponibilità al giorno di materia prima A è pari a 6 ton, mentre quella di materia prima B è di 8 ton. La quantità di A e B consumata per produrre una ton di vernice E ed I è riportata nella seguente tabella.
											
											\quad\qquad\qquad\qquad Vernici\\
						Materie prime
						\begin{tabular}{lll}
								& E & I \\
							A 	& 1	& 2 \\
							B	& 2	& 1
						\end{tabular}

					}
					{\begin{itemize}
						\item $3k\$$ per E.
						\item $2k\$$ per I.
						\item Disponibilità A 6 tonnellate.
						\item Disponibilità B 8 tonnellate.
					\end{itemize}}
					{\begin{itemize}
						\item $x_E$ Tonnellate vernice E.
						\item $x_I$ Tonnellate vernice I.
					\end{itemize}}
					{\begin{center}
					\end{center}}
					{$$\max 3x_E + 2x_I$$
					$$1x_E + 2x_I \leq 6$$
					$$2x_E + 1x_I \leq 8$$
					$$x_I \leq x_E + 1$$
					$$x_I \leq 2$$
					$$x_E, x_I \geq 0 $$
					}
			\paragraph{Esempio problema dieta}
				\problema{Quali ingredienti e in che quantità miscelare per minimizzare il costo del mangime.}
					{Ogni dose deve contenere \textit{almeno} 2hg di proteine, 4hg di carboidrati, 3hg di grasso.}
					{$x_j$ = kg ingredienti di $j$.}
					{\begin{tabular}{lllll}
						
						Ingrediente & Proteine & Carboidrati & Grasso & Costo $\$$/kg\\
						1 & 1 & 4 & 3 & 3\\
						2 & 3 & 4 & 2 & 6\\
						3 & 2 & 3 & 3 & 5\\
						4 & 2 & 2 & 4 & 6\\
					\end{tabular}}
					{$$\min 3x_1 + 6x_2 + 5x_3 + 6x_4$$
					$$1x_1 + 3x_2 + 2x_3 + 2x_4 \geq 2 \text{ proteine}$$
					$$4x_1 + 4x_2 + 3x_3 + 2x_4 \geq 4 \text{ carboidrati}$$
					$$3x_1 + 2x_2 + 3x_3 + 4x_4 \geq 3 \text{ grasso}$$
					$$x_j \geq 0$$
					}
						
\end{document}
