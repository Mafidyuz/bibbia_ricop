\documentclass[12pt, twoside, letterpaper]{article}
\usepackage[top=2cm,bottom=4cm,left=3cm,right=3cm,asymmetric]{geometry}% aggiungere ^twoside^
\usepackage{color}   %May be necessary if you want to color links
\usepackage{hyperref}
\hypersetup{
    colorlinks=true, %set true if you want colored links
    linktoc=all,     %set to all if you want both sections and subsections linked
    linkcolor=black,  %choose some color if you want links to stand out
}
\usepackage[utf8x]{inputenc}
%\usepackage[table]{xcolor}
\usepackage[english]{babel}
\usepackage{amsmath, amsthm, amssymb, amsfonts}
\usepackage[breakable]{tcolorbox}
\usepackage{floatrow}
\usepackage{caption}
\newtcolorbox{dati}[1][]{colback=green!30!white, bottomtitle=1.5mm,breakable,#1}
\newtcolorbox{variabili}[1][]{colback=red!30!white, bottomtitle=1.5mm,breakable,#1}
\newtcolorbox{vincoli}[1][]{colback=orange!30!white, bottomtitle=1.5mm,breakable,#1}
\newtcolorbox{obiettivo}[1][]{colback=yellow!30!white, bottomtitle=1.5mm,breakable,#1}
\newenvironment{rcases}
  {\left.\begin{aligned}}
  {\end{aligned}\right\rbrace}
\newcommand{\casi}[3]{
	$#1 = \begin{cases} \text{#2} \\ \text{#3} \end{cases}$
}
\newcommand{\vc}[0]{
	\underline{c}
}
\newcommand{\vp}[0]{
	\underline{P}
}
\newcommand{\vx}[0]{
	\underline{x}
}
\newcommand{\va}[0]{
	\underline{a}
}
\newcommand{\vy}[0]{
	\underline{y}
}
\newcommand{\vz}[0]{
	\underline{z}
}
\newcommand{\fx}[0]{
	f(\underline{x})
}
\newcommand{\fy}[0]{
	f(\underline{y})
}
\newcommand{\fz}[0]{
	f(\underline{z})
}
\newcommand{\img}[3] {
	\begin{figure}[h]
		\caption*{#1}
		\centering
		\includegraphics[scale=#2]{img/#3}\\
	\end{figure}
}
\newcommand{\problema}[5]{
	#1
	\begin{dati}
		\paragraph{Dati} #2
	\end{dati}
	\begin{variabili}
		\paragraph{Variabili} #3
	\end{variabili}
	\begin{obiettivo}
		\paragraph{Obiettivo} #4
	\end{obiettivo}
	\begin{vincoli}
		\paragraph{Vincoli} #5
	\end{vincoli}
}
\newcommand{\problemai}[5]{
	#1
	\begin{dati}
		\paragraph{Dati} 
			\begin{itemize}
				#2
			\end{itemize}
	\end{dati}
	\begin{variabili}
		\paragraph{Variabili} 
			\begin{itemize}
				#3
			\end{itemize}
	\end{variabili}
	\begin{obiettivo}
		\paragraph{Obiettivo} 
			\begin{itemize}
				#4
			\end{itemize}
	\end{obiettivo}
	\begin{vincoli}
		\paragraph{Vincoli}
			\begin{itemize}
				#5
			\end{itemize}
	\end{vincoli}
}

\title{Ricerca operativa}
\author{Mario Petruccelli \cr Università degli studi di Milano}
\date{A.A. 2019/2020}

\addto\captionsenglish{% Replace "english" with the language you use
  \renewcommand{\contentsname}%
    {Sommario}%
}

\begin{document}

	\begin{titlepage}
		\maketitle
		\newpage
		\tableofcontents
	\end{titlepage}


	\section{Introduzione}
	
		\textbf{Ricerca operativa:} disciplina che affronta la risoluzione di problemi decisionali complessi tramite 	modelli matematici e algoritmi.
	Si parte da un \textbf{sistema organizzato} e lo si formalizza in un \textbf{modello matematico} per poi risolverlo tramite \textbf{algoritmi}.
		\subsection{Tassonomia modelli}
		\begin{itemize}
			\item \textbf{Descrittivi} $\rightarrow$ Modelli che cercano di descrivere o simulare sistemi complessi \textit{(e.g. modellini, plastici, \dots)}
			\item \textbf{Predittivi} $\rightarrow$ Modelli che cercano di predire dei dati \textit{(e.g. andamento mercati finanziari, previsioni, \dots)}
			\item \textbf{Prescrittivi} $\rightarrow$ Modelli che trovano la soluzione ottimale ad un problema \textit{(sono quelli che studieremo in questo corso)}. 
			\\\\La descrizione del problema avverrà attraverso \textbf{vincoli}, \textbf{obiettivi}.
		\end{itemize}
		\paragraph{Esempio di problemi decisionali}
		\begin{itemize}
			\item Finanza (investimenti)
			\item Produzione (dimensionamento, organizzazione, \dots)
			\item Logistica (gestione scorte, quanta merce, \dots)
			\item Gestione (pianificazione, turnistica personale, \dots)
			\item Servizi (rotte, \dots)
		\end{itemize}
		\paragraph{NB}\textit{Lo stesso modello può servire per risolvere problemi diversi.}
		\\\\\textbf{Set covering }Problema per la gestione di un territorio. I problemi dei sismografi e dei ripetitori sono diversi ma si ragiona allo stesso modo.
		\subsection{Programmazione matematica} La programmazione matematica (intesa come \textit{pianificazione} delle azioni necessarie per individuare la soluzione ottima) è ciò che rappresenta il processo risolutivo nella ricerca operativa: 
			\begin{itemize}
				\item Analisi del problema e scrittura di un modello matematico.
				\item Definizione e applicazione di un metodo di soluzione.
			\end{itemize}
		In particolar modo, la programmazione matematica si occupa di ottimizzare una funzione di più variabili, spesso soggette a dei vincoli. A seconda del tipo di modello abbiamo:
		\begin{itemize}
			\item Programmazione lineare continua.
			\item Programmazione lineare intera.
			\item Programmazione booleana.
		\end{itemize}
	
	\section{Modellazione di problemi}
	
		\subsection{Problema dello zaino} 
				\problema{
					Ci sono $n$ oggetti di valore $p_j$ e ingombro $w_j$ per $j=1, \dots, n$ ed è data la capacità massima $b$ di un contenitore.
			\paragraph{Problema} Quali oggetti inserire nel contenitore senza superare capacità. 
			\paragraph{Obiettivo}Massimizzare il valore degli oggetti. Si tratta di un problema di \textbf{ottimizzazione} e va formalizzato in modello matematico. Ci sono 4 componenti fondamentali.}
				{I dati sono informazioni conosciute a priori, in questo caso sono:
				\begin{itemize}
					\item $p_j \rightarrow$ valore dell'oggetto $j$.
					\item $w_j \rightarrow$ ingombro dell'oggetto $j$.
					\item $b \rightarrow$ capacità massima del contenitore.
				\end{itemize}}
				{Le variabili sono elementi che rappresentano una decisione. 
				\begin{itemize}
					\item $x_j= \begin{cases} \text{1 se il j-esimo oggetto viene inserito} \\ \text{0 altrimenti}\end{cases}$
				\end{itemize}}
				{L'obiettivo è la funzione che rappresenta il risultato da ottenere.
				\begin{itemize}
					\item $max \sum_{j=1}^np_jx_j \rightarrow \textit{massimizzare il valore}$
				\end{itemize}				  }
				{I vincoli sono le limitazioni che abbiamo sui dati. 
				\begin{itemize}
					\item $\sum_{j=1}^n w_j x_j \leq b \rightarrow$ la somma degli ingombri degli oggetti presi non può superare la capacità del contenitore 
					 \item $x_j \in \{0,1\} \quad j=1, \dots, n$
				\end{itemize}}
		
		\subsection{Problema di trasporto e localizzazione di impianti}
			\problema
			{Ci sono $n$ siti candidati ad ospitare unità produttive, ciascuno con capacità massima $a_i$ con $i=1, \dots, n$. Vi sono $m$ magazzini, ognuno con una domanda da soddisfare $b_j$ con $j = 1, \dots, m$. Indichiamo con $c_{ij}$ il costo di trasporto di una unità di prodotto dal sito $i$ al magazzino $j$. L'attivazione di una unità produttiva nel sito $i$ ha un costo fisso $f_i$. 
			\paragraph{Problema} Dove aprire le unità produttive e come trasportare il prodotto dalle unità produttive aperte ai magazini in modo da soddisfare la domanda.
			\paragraph{Obiettivo} Minimizzare i costi di apertura e trasporto.}
			{\begin{itemize}
				\item $a_i \rightarrow$ capacità di produzione del sito $i$
				\item $b_j \rightarrow$ domanda del magazzino $j$ 
				\item $c_{ij} \rightarrow$ costo del trasporto di un'unità dal sito $i$ al magazzino $j$. 
				\item $f_i \rightarrow$ costo di attivazione unità nel sito $i$.  
			\end{itemize}} 
			{\begin{itemize} 
				\item $y_i = \begin{cases} \text{1 se il sito $i$ ospita un'unità produttiva} \\ \text{0 altrimenti}\end{cases}$
				\item $x_{ij} =$ numero di unità trasportata dal sito $i$ al magazzino $j$.
			\end{itemize}}
			{\begin{itemize}
				\item $min \sum_{i=1}^n \sum_{j=1}^m c_{ij}x_{ij} + \sum_{i=1}^n f_i y_i \rightarrow$ minimizzare il costo di attivazione di un unità nei vari siti e il costo dei trasporti delle unità.
			\end{itemize}}
			{\begin{itemize}
				\item $\sum_{j=1}^m x_{ij} \leq a_i y_i \quad i = 1, \dots, n \rightarrow$ le unità trasportate da un sito i non possono superare la capacità $a_i$ di quel sito $i$.
				\item $\sum_{i=1}^n x_{ij} \geq b_j \quad j=1, \dots, m \rightarrow$ Le unità inviate ad un magazzino $j$ dai vari siti deve soddisfare la domanda di quel magazzino.
				\item $x_{ij} \geq 0 \quad i=1, \dots, n \quad j = 1, \dots, m$
				\item $y_i \in \{0,1\} \quad i=1, \dots, n$
			\end{itemize}}
						
		\subsection{Problema assegnamento}					
			\problemai
			{Ci sono $n$ lavoratori e $n$ attività. Indichiamo con $t_{ij}$ il tempo impiegato dal lavoratore $i$ per svolgere l'attività $j$.
			\paragraph{Problema} Assegnare a ciascun lavoratore una sola attività, così che tutte le attività siano svolte.
			\paragraph{Obiettivo} Minimizzare il tempo richiesto a svolgere l'attività $j$.}
			{\item $t_{ij} \rightarrow$ tempo impiegato dal lavoratore $i$ per svolgere l'attività $j$.}
			{\item $x_{ij} = \begin{cases} \text{1 se il lavoratore $i$ svolge l'attività $j$} \\ \text{0 altrimenti}\end{cases}$ }
			{\item $min \sum_{i=1}^n\sum_{j=1}^n t_{ij} x_{ij} \rightarrow$ minimizzare il tempo speso per svolgere tutte le attività dei vari lavoratori.}
			{\item $\sum_{j=1}^n x_{ij} = 1 \quad \forall i \rightarrow$ a ogni lavoratore è associata una sola attività.
			\item $\sum_{i=1}^n x_{ij} = 1 \quad \forall j \rightarrow$ a ogni attività è associata nn solo lavoratore.
			\item $x_{ij} \in \{0,1\} \quad \forall i,j$}

		
		\subsection{Mix Produttivo}
			\problemai
				{Si hanno $m$ risorse produttive con disponibilità $b_i$. Si possono produrre $n$ prodotti diversi. Per produrre una unità di un prodotto \textit{j-esimo} si utilizzano $a_{ij}$ unità della risorsa \textit{i-esima}. Ciascun prodotto ha un profitto unitario $c_j$.}
				{\item $b_i \rightarrow$ disponibilità risorsa \textit{i-esima}.
				\item $a_{ij} \rightarrow$ unità della risorsa \textit{i-esima} usate per produrre un prodotto \textit{j-esimo}.
				\item $c_j \rightarrow$ profitto di un unità del prodotto $j$. }
				{\item $x_j =$ unità prodotte del prodotto \textit{j-esimo}.}
				{\item $max \sum_{j=1}^n c_j x_j \rightarrow$ massimizzare il profitto tra i vari prodotti.}
				{\item $\sum_{j=1}^n a_{ij} x_j \leq b_i \quad \forall i \rightarrow$ le risorse usate nella produzione non possono superare la disponibilità di ciascuna risorsa.
				\item $x_j \geq 0 \quad \forall j$}


			\subsubsection{Vernici}
				\problema
					{L'azienda produce due tipi di vernici, una vernice per interni (\texttt{I}) e una venrice per esterni (\texttt{E}), usando due materie prime indicate con \texttt{A} e \texttt{B}. La disponibilità al giorno di materia prima \texttt{A} è pari a \texttt{6 ton}, mentre quella di materia prima \texttt{B} è di \texttt{8 ton}. La quantità di \texttt{A} e \texttt{B} consumata per produrre una \texttt{ton} di vernice \texttt{E} ed \texttt{I} è riportata nella seguente tabella.
											
											\quad\qquad\qquad\qquad Vernici\\
						Materie prime
						\begin{tabular}{lll}
								& E & I \\
							A 	& 1	& 2 \\
							B	& 2	& 1
						\end{tabular}
						
					Si ipotizza che tutta la vernice prodotta venga venduta. Il prezzo di vendita per tonnellata è \texttt{3K\$} per \texttt{E} e \texttt{2Ks} per \texttt{I}. L'azienda ha effettuato un'indagine di mercato con i seguenti esisti: 
					\begin{itemize}
						\item La domanda giornaliera di vernice \texttt{I} non super mai di più di \texttt{1 ton} quella di vernice \texttt{E}.
						\item La domanda massima giornaliera di vernice \texttt{I} è di \texttt{2 ton.}
					\end{itemize}
					}
					{\begin{itemize}
						\item \texttt{3k\$} per \texttt{E}.
						\item \texttt{2k\$} per \texttt{I}.
						\item Disponibilità \texttt{A} 6 tonnellate.
						\item Disponibilità \texttt{B} 8 tonnellate.
					\end{itemize}}
					{\begin{itemize}
						\item $x_E$ Tonnellate vernice E.
						\item $x_I$ Tonnellate vernice I.
					\end{itemize}}
					{\begin{itemize}
						\item $\max 3x_E + 2x_I$
					\end{itemize}}
					{\begin{itemize}
						\item $x_E + 2x_I \leq 6$
						\item $2x_E + x_I \leq 8$
						\item $x_I - x_E \leq 1$
						\item $x_I \leq 2$
						\item $x_E, x_I \geq 0 $
					\end{itemize}}
			\subsubsection{Problema della dieta}
				\problemai{Un determinato mangime per animali deve contenere in ogni dose almeno \texttt{2hg} di proteine, \texttt{4hg} di carboidrati e \texttt{3hg} di grasso. Si possono miscelare 4 ingredienti con le seguenti caratteristiche (\textit{in hg per ogni kg)}.\\\\
				\begin{tabular}{lllll}
					Ingrediente & Proteine & Carboidrati & Grasso & Costo euro/kg\\					1 & 1 & 4 & 3 & 3\\
					2 & 3 & 4 & 2 & 6\\
					3 & 2 & 3 & 3 & 5\\
					4 & 2 & 2 & 4 & 6\\
				\end{tabular}
				\paragraph{Problema} Determinare quali ingredienti ed in quale quantità miscelare in modo da minimizzare il costo del mangime.} 
					{\item Ogni dose deve contenere \textit{almeno} \texttt{2hg} di proteine, \texttt{4hg} di carboidrati, \texttt{3hg} di grasso.}
					{\item $x_j$ = quantità di ingredienti $j$ in \texttt{kg}.}
					{\item $min \sum_{j=1}^4 c_jx_j \rightarrow$ minimizzare il costo.}
					{\item $\min 3x_1 + 6x_2 + 5x_3 + 6x_4$
					\item $x_1 + 3x_2 + 2x_3 + 2x_4 \geq 2hg \text{ proteine}$
					\item $4x_1 + 4x_2 + 3x_3 + 2x_4 \geq 4hg \text{ carboidrati}$
					\item$3x_1 + 2x_2 + 3x_3 + 4x_4 \geq 3hg \text{ grasso}$
					\item $x_j \geq 0 \quad \j = 1 \dots 4$ 
					}
		
		\subsection{Miscelazione}			
			\problemai
			{
				Due tipi di benzina si ottengono miscelando 3 tipi di materie grezze. Le due benzine sono vendute rispettivamente a  40 cent/l e a 30 cent/l. Le materie grezze sono vendute a 10 cent/l, 16 cent/l, 14 cent/l e sono disponibili in quantità giornaliere pari a 100000 l, 70000 l, 120000 l.
				\paragraph{Problema} Produrre benzina con le quantità di materie a disposizione.
				\paragraph{Obiettivo} Massimizzare il profitto tenendo conto del costo delle materie.
			}
			{
				\item $r_j \rightarrow$ ricavo benzina \textit{j-esima}.
				\item $c_i \rightarrow$ costo petrolio \textit{i-esimo}.
				\item $a_i \rightarrow$ quantità giornaliera di petrolio \textit{i-esimo}.
				\item $\%_{ij}^{m/M} \rightarrow$ percentuale minima e massima di petrolio \textit{i-esimo} da avere all'interno della benzina \textit{j-esima}.
			}
			{
				\item $p_{ij} = $ percentuale di petrolio $i$ in benzina $j$.
				\item $b_j =$ litri di benzina \textit{j-esima} prodotti.
				\item $p_i = $ litri di petrolio \textit{i-esimo} usati.
				\item $x_{ij} := p_{ij} b_j \rightarrow$ litri di petrolio \textit{i-esimo} usato per produrre i litri di benzina \textit{j-esima} \textit{(ciò permette di ottenere un modello lineare)}.
			}
			{
				\item $max [ \underbrace{\sum_{j=1}^n(\sum_{i=1}^m x_{ij}) r_j}_{\text{litri prodotti di benzina \textit{j-esima}}} - \underbrace{\sum_{i=1}^m(\sum_{j=1}^nx_{ij}) c_i}_{\text{litri di petrolio \textit{i-esimo utilizzati}}}] \rightarrow$ massimizzare il ricavo netto della produzione.
			}
			{
				\item $\sum_{j=1}^n x_{ij} \leq a_i \quad \forall i \rightarrow$ i litri di petrolio \textit{i-esimo} utilizzati non possono superare i litri disponibili giornalmente.
				\item $\%_{ij}^m (\sum_{k=1}^m x_{kj}) \leq x_{ij} \leq \%_{ij}^M (\sum_{k=1}^m x_{kj}) \quad \forall i,j \rightarrow$ i litri di petrolio \textit{i-esimo} all'interno della miscela per benzina \textit{j-esima} deve essere compresa tra gli estremi di percentuale dati dalla tabella.
				\item $x_{ij} \geq 0 \quad \forall i,j$
			}
			
		\subsection{Turnazione personale}
			
			\problemai{
				Ci sono 3 turni lavorativi (mattina, pomeriggio, notte). Sono presenti $n$ lavoratori che svolgono 5 turni settimanali, dopo un turno lavorativo per un lavoratore ce ne devono essere almeno 2 di riposo. Ogni lavoratore propone 5 turni in ordine di preferenza.
			
				\paragraph{Problema} Organizzare turni in modo tale che ognuno si coperto.
				\paragraph{Obiettivo} Minimizzare il grado di soddisfacibilità globale.
			}{
				\item $p_{gt}^j \rightarrow$ grado di soddisfacibilità del lavoratore \textit{j-esimo} a lavorare il giorno $g$ nel turno $t$.
				\item $r_{gt} \rightarrow$ numero lavoratori necessari il giorno $g$ al turno $t$.
			}{
				\item $x_{gt}^j = \begin{cases} \text{1 se il lavoratore \textit{j-esimo} lavora il giorno $g$ al turno $t$} \\ \text{0 altrimenti} \end{cases}$
			}{
				\item $min \sum_{j=1}^n \sum_{g \in G} \sum_{t \in T} p_{gt}^j x_{gt}^j$
			}{
				\item $\sum_{j=1}^n x_{gt}^j \geq r_{gt} \quad \forall g,t \rightarrow$ per ogni turno giornaliero ci devono essere tanti lavoratori quanti sono richiesti.
				\item $\sum_{g \in G} \sum_{t \in T} x_{gt}^j \geq 5 \quad \forall j \rightarrow$ ogni lavoratore deve lavorare per almeno 5 turni.	
				\item 	\begin{equation*}
							\begin{rcases}
								x_{l,m}^j + x_{l,p}^j + x_{l,s}^j \leq 1 \\
								x_{l,p}^j + x_{l,s}^j + x_{ma,m}^j \leq 1 \\
								\dots \\
								x_{d,m}^j + x_{d,p}^j + x_{d,s}^j \leq 1 \\
							\end{rcases}
							\substack{\text{ $\forall j \rightarrow$ ogni lavoratore deve avere almeno 2 turni di riposo dopo un } \\ \text{turno di lavoro.}}
						\end{equation*} 

			}
			
			
		\subsection{Locazione di servizi}
			\problemai
			{
				Abbiamo un insieme $N = \{1, \dots, n\}$ di potenziali localizzazioni di servizi ed un insieme $I = \{1, \dots, m\}$ di clienti. Ogni località $j$ ha una capacità $u_j$ e costo di attivazione $c_j$. Ogni cliente $i$ ha una richiesta $b_i$. Il costo da sostenere per servire il cliente $i$ dalla località $j$ è $h_{ij}$ e ogni cliente è servito da una sola località.
				
				\paragraph{Problema} Determinare la localizzazione di servizi così da soddisfare ogni cliente.
				\paragraph{Obiettivo} Minimizzare il costo di attivazione e di servizio complessivo.
			}
			{
				\item $u_j \rightarrow$ capacità località \textit{j-esima}.
				\item $c_j \rightarrow$ costo attivazione località \textit{j-esima}.
				\item $b_i \rightarrow$ richiesta cliente \textit{i-esimo}.
				\item $h_{ij} \rightarrow$ costo per servire il cliente $i$ dalla località $j$.
			}
			{
				\item $y_j = \begin{cases} \text{1 se è attivo un servizio nella località \textit{j-esima}} \\ \text{0 altrimenti} \end{cases}$
				\item $x_{ij} = \begin{cases} \text{1 se il cliente \textit{i} viene servito dalla località \textit{j}} \\  \text{0 altrimenti} \end{cases}$
			}
			{
				\item $min [ \sum_{j \in N} \sum_{i \in I} h_{ij}x_{ij} + \sum_{j \in N} c_jy_j] \rightarrow$ minimizzare il costo di attivazione di servizio tra i vari siti e clienti.
			}
			{
				\item $\sum_{j \in N} x_{ij} = 1 \quad \forall i \in I \rightarrow$ ogni cliente $i$ è servito da una sola località $j$.
				\item $\sum_{i \in I} x_{ij} b_i \leq u_j \quad \forall j \in N \rightarrow$ ogni località $j$ attiva deve soddisfare la domanda $b_i$ del cliente $i$ associato.
				\item $x_{ij} \leq y_j \quad \forall i \in I, j \in N$
				\item $x_{ij} \in \{0,1\} \quad \forall i,j$
				\item $y_j \in \{0,1\} \quad \forall j$ 
				
				Gli ultimi 3 sono vincoli opzionali (ma non troppo). Il simplesso risolve il modello nel continuo, da cui tira poi fuori la soluzione intera.
			}
		
		\subsection{Bin packing}
			\problemai{Ci sono $n$ oggetti, ciascuno con ingombro $w_j$. Sono dati dei contenitori di capacità $b$.
			
			\paragraph{Problema} Assegnare gli oggetti ai contenitori rispettando le capacità.
			\paragraph{Obiettivo} Minimizzare il numero di contenitori usati.
			}{
				\item $w_j \rightarrow$ ingombro oggetto \textit{j-esimo}.
				\item $b$ capacità contenitori. 
			}{
				\item $x_{ij} = \begin{cases} \text{1 se il contenitore \textit{i-esimo} accetta l'oggetto \textit{j-esimo}} \\ \text{0 altrimenti} \end{cases}$ 
				\item $y_i = \begin{cases} \text{1 se uso il contenitore \textit{i-esimo}} \\ \text{0 altrimenti} \end{cases}$ 
			}{
				\item $min \sum_{i=1}^n y_i \rightarrow$ minimizzare il numero di contenitori usati.
			}{
				\item $\sum_{j=1}^n x_{ij} w_j \leq b y_i \quad \forall i \rightarrow$ tutti gli oggetti contenuti in ogni contenitore \textit{i-esimo} non devono superare la capacità $b$.
				\item $\sum_{i=1}^n x_{ij} = 1 \quad \forall j \rightarrow$ ogni oggetto può essere messo in un unico contenitore.
				\item $x_{ij} \in \{0,1\} \quad \forall i,j$
				\item $y_i \in \{0,1\} \quad \forall i$
			}
		
		\subsection{Problema di assegnamento}
			\problemai{
				Ci sono $m$ macchine identiche e $n$ lavorazioni. Ogni lavorazione $j$ richiede di essere processata da una qualsiasi delle $m$ macchine per una durata ininterrotta $p_j$. Ogni macchina processa una sola lavorazione alla volta.
				
				\paragraph{Problema} Come assegnare le lavorazioni alle macchine.
				\paragraph{Obiettivo} Minimizzare l'istante di completamento della macchina che termina per ultima.
			}{
				\item $p_j \rightarrow$ durata della lavorazione \textit{j-esima}.
			}{	
				\item $x_{ij} = \begin{cases} \text{1 se la macchina \textit{i-esima} effettua la lavorazione $j$} \\ \text{0 altrimenti} \end{cases}$
				\item $T \geq 0 \rightarrow$ istante di completamento della macchina che termina per ultima.
			}{
				\item $min$ $max \sum_{i=1}^n \sum_{j=1}^m x_{ij} p_j \rightarrow$ minimizzare l'istante di completamento della macchina che termina per ultima.
			}{
				\item $\sum_{j=1}^m x_{ij}p_j \leq T \quad \forall i \rightarrow$ ogni macchina termina le lavorazioni al più in contemporanea con la macchna che termina per ultima.
				\item $\sum_{i=1}^n x_{ij} = 1 \quad \forall j \rightarrow$ ogni lavorazione \textit{i-esima} è effettuata da na e una sola macchina.
				\item $x_{ij} \in \{0,1\} \quad \forall i,j$
			}
			
		\subsection{Problema di sequenziamento monoprocessore}
			\problemai{
				C'è una macchina e ci sono $n$ lavorazioni. Ogni lavorazione $j$ ha un tempo di processamento $p_j$, è disponibile a partire dall'istante $r_j$ e deve essere completata entro la data $d_j$. La macchina può processare una sola lavorazione alla volta. 
				
				\paragraph{Problema} In quale ordine processare le lavorazioni sulla macchina.
				\paragraph{Obiettivo} Minimizzare la somma degli istanti di completamento di tutte le lavorazioni.
			}{
				\item $p_j \rightarrow$ tempo di processamento della lavorazione \textit{j-esima}.
				\item $r_j \rightarrow$ istante minimo di inizio della lavorazione \textit{j-esima}.
				\item $d_j \rightarrow$ istante massimo di completamento della lavorazione \textit{j-esima}.
			}{
				\item \casi{x_{ij}}{1 se la lavorazione $i$ precede la lavorazione $j$}{0 altrimenti}
				\item $c_j$ = istante di completamento del lavoro \textit{j-esimo}.
			}{
				\item $min \sum_{j=1}^n c_j \rightarrow$ minimizzare la somma degli istanti di completamento delle lavorazioni.
			}{
				\item $c_j \leq c_k - p_k + M (1-x_{jk}) \quad 1 \leq j < k \leq n \quad M \in R \rightarrow$ se $j$ precede $k$, il suo istante di completamento al più coincide con quello di inizio di $k$.
				\item $c_k \leq c_j - p_j + M x_{jk} \quad 1 \leq j < k \leq n \quad M \in R \rightarrow$ vincolo ridondante, poichè tiene conto del caso opposto ($k$ precede $j$).
				\item $0 \leq p_j + r_j \leq c_j \leq d_j \quad \forall j = 1, \dots, n \rightarrow$ l'istante minimo di fine processo $j$ $(p_j+r_j)$ al più coincide con il suo istante di completamento effettivo, ed al più coincide con l'istante massimo di completamento.
				\item $x_{jk} \in \{0,1\} \quad 1 \leq j < k \leq n$
			}
			
		\subsection{Problema di pianificazione della produzione}
			\problemai{
				Determiniamo un piano di produzione di un prodotto specifico nell'arco di $n$ periodi. Per ciascun periodo conosciamo la domanda da soddisfare $d_t$, il costo di produzione $c_t$ e il costo di magazzino $i_t$ per unità di prodotto. La capacità massima di produzione è $c$.
				
				\paragraph{Problema} Pianificare la produzione così da soddisfare la domanda per ogni periodo.
				
				\paragraph{Obiettivo} minimizzare i costi.
			}{
				\item $d_t \rightarrow$ domanda del periodo \textit{t-esimo}.
				\item $c_t \rightarrow$ costo di produzione di un'unità nel periodo \textit{t-esimo}.
				\item $i_t \rightarrow$ costo di magazzino di un unità nel periodo \textit{t-esimo}.
				\item $c \rightarrow$ capacità massima di produzione.
			}{
				\item $x_t =$ unità del prodotto nel periodo \textit{t-esimo}.
				\item $m_t =$ unità del prodotto immagazzinate al termine del periodo \textit{t-esimo}
			}{
				\item $min \sum _{t=1}^n c_t x_t + \sum_{t=1}^n i_t m_t \rightarrow$ minimizzare il costo di produzione e di magazzino.
			}{
				\item $m_{t-1} + x_t = d_t + m_t \quad t = 1, \dots, n \rightarrow$ le unità immagazzinate dal periodo precedente insieme alle unità prodotte nel periodo $t$ attuale devono coincidere con la domanda nel periodo $t$ sommati ai prodotti rimanenti immagazzinati.
				\item $x_t \leq c \quad t = 1, \dots, n \rightarrow$ le unità prodotte non possono superare la capacità produttiva.
			}
			Questo problema è rappresentabile tramite un \textbf{modello di flusso}:
			\img{Modello di flusso}{0.7}{img1.png}
			\\A ogni periodo la produzione deve soddisfare la domanda (e non superare la capacità) e le unità rimaste vanno in magazzino.	
			
			\subsubsection{Variante lotto minimo}
				Ogni periodo il lotto minimo è pari a $L$.
				\paragraph{Variabili} \casi{y_t}{1 se produco un lotto al periodo \textit{t-esimo}}{0 altrimenti}
				
				\paragraph{Vincoli} 
					\begin{itemize}
						\item $x_t \geq L y_t \quad t=1, \dots, n$
						\item $x_t \leq c y_t \quad t=1, \dots, n$
					\end{itemize}
					
			\subsubsection{Variante produzione con costi fissi}
				Se in un periodo è stata prodotta almeno una unità di prodotto si aggiunge un costo fisso $k$.
				\paragraph{Obiettivo}
					$min \sum_{t=1}^n c_t x_t + \sum_{t=1}^n i_t m_t + \sum_{t=1}^n k y_c \rightarrow$ aggiungo il costo fisso nel caso in cui produco qualcosa.
			
			\subsubsection{Variante multiprodotto}
				Possono essere fabbricati più prodotti durante gli $n$ periodi.
				
				\paragraph{Variabili}
					\begin{itemize}
						\item $x_t^j \rightarrow$ unità di prodotto $j$ fabbricate nel peridodo $t$.
						\item $m_t^j \rightarrow$ unità di prodotto $j$ immagazzinate nel periodo $t$.						
					\end{itemize}
					
				\paragraph{Obiettivo}
					$min \sum_j(\sum_{t=1}^n c_t^j x_t^j + \sum_{t=1}^n i_t^j m_t^j + \sum_{t=1}^n ky_t^j)$
		
		\subsection{Set Covering}
			\problemai{
				È dato un insieme $M=\{1,2, \dots, m\}$ ed una famiglia di $n$ suoi sottoinsiemi $S_j \subseteq M$. Ogni sottoinsieme ha un costo $c_j$
				
				\paragraph{Problema} Trovare un insieme $T \subseteq N = \{1, \dots, n\}$ tale che l'unione degli $S_j$, con $j \in T$ sia uguale a $M$.
				
				\paragraph{Obiettivo} Minimizzare i costi dei sottoinsiemi scelti.
			}{
				\item $M \rightarrow$ insieme di partenza.
				\item $S_j \rightarrow$ sottoinsieme di $M$.
				\item $c_j \rightarrow$ costo del sottoinsieme \textit{j-esimo}.
			}{
				\item \casi{y_j}{1 se scelgo il sottoinsieme \textit{j-esimo} $S_j$.}{0 altrimenti.}
			}{
				\item $min \sum_{j \in N} c_j y_j \rightarrow$ minimizzare il costo dei  sottoinsiemi scelti per coprire $M$.
			}{
				\item $\bigcup_{j \in T} S_j = M \rightarrow$ l'unione dei sottoinsiemi scelti coincide con $M$.\\
				Tuttavia $S_j$ è rappresentabile come un vettore di $m$ elementi:
				\item $S_j = \begin{pmatrix}
					a_{1,j} \\
					a_{2,j} \\
					\dots \\
					a_{m,j}
				\end{pmatrix} \rightarrow$ \casi{a_{ij}}{1 se l'elemento \textit{i-esimo} appartiene al sottoinsieme $S_j$}{0 altrimenti.}
				\\Inoltre, il sottoinsieme $S_j$ viene viene scelto se la rispettiva variabile $y_j$ è uguale a 1, quindi utilizziamo un vettore di $n$ elementi, costituito da $y_j$.\\
				\underline{\textit{Y}}$ = \begin{pmatrix}
					y_1 \\
					y_2 \\
					\dots \\
					y_n
				\end{pmatrix}$ 
				\\\\Ora rappresentiamo anche $a_i,j$ sotto forma matriciale:\\\\
				$A = \begin{pmatrix}
					a_{1,1} & a_{1,2} & \dots & a_{1,n} \\
					\dots & \dots & \dots & \dots \\
					a_{m,1} & a_{m,2} & \dots & a_{m,n}
				\end{pmatrix}$\\\\
				Se effettuiamo il prodotto matriciale tra $A$ e \underline{\textit{Y}} otteniamo un sistema, in cui ogni riga rappresenta quante volte il valore $i \in M$ compare tra tutti i sottoinsiemi scelti.\\\\
				$ A \underline{Y} \geq 1 \rightarrow \begin{cases} a_{1,1} y_1 + a_{1,2}y_2 + \dots + a_{1,n} y_n \geq 1 \\ \dots \\ a_{m,1} y_1 + a_{m,2}y_2 + \dots + a_{m,n} y_n \geq 1 \end{cases}$
				\item $y_j \in \{0,1\} \quad \forall j \rightarrow \underline{Y} \in \{0,1\}^n$ 
			}
			
		\newpage
		\subsection{Modellare vincoli logici utilizzando variabili binarie}
			In primo luogo associamo una variabile binaria a ciascuna variabile logica. Ad esempio alla variabile logica $X=$"attivare l’impianto di produzione" associamo la variabile binaria $x$ nel seguente modo:
			\casi{x}{1 "attivazione dell'impianto di produzione"}{0 "non attivazione dell'impianto di produzione"}
			\begin{tabular}{ll}
				Rappresentiamo & con\\
				$\lnot X$ & $(1-x)$\\	
				$X \lor Y$ & ($x+y$)\\
				$X \rightarrow Y$ & $x \leq y$\\
				$X_1 \lor X_2 \lor \dots \lor X_n \rightarrow Y$ & $\sum_1^n x_i \leq ny$\\
				$X \rightarrow Y_1 \lor Y_2 \lor \dots \lor Y_n$ & $x \leq \sum_1^n y_i$\\
				$X \rightarrow Y_1 \land Y_2 \land \dots \land Y_n$ & $nx \leq \sum_1^n y_i$
			\end{tabular}\\\\\\
			\begin{tabular}{lll}
				Esempi & & \\
				$X \rightarrow (\lnot Y \lor \lnot Z)$ & diviene & $x \leq (1-y)+(1-z) \quad $ cioè $\quad x+y+z \leq 2$\\
				$(X \lor Y) \rightarrow (\lnot Z)$ & diviene & $x+y \leq 2(1-z)$\\
				$(X \lor Y) \land (\lnot Z \lor Y)$ & diviene & $x+y \geq 1, (1-z)+y \geq 1$\\
				Almeno due fra $X,Y,Z$ & diviene & $x+y+z \geq 2$\\
				Al più $k$ fra $X_1, X_2, \dots, X_n$ & diviene & $\sum_1^n x_i \leq k$
			\end{tabular}
			\newpage
			
	\section{Parte formale della ricerca operativa}
		\subsection{Tecnica di soluzione lineare}
			Prendiamo un esempio che abbiamo già visto, il problema delle vernici.
			\paragraph{Obiettivo} 
				\begin{itemize} 
					\item $\max 3x_E + 2x_I \leq 6$ 
				\end{itemize}
				
			\paragraph{Vincoli}
				\begin{itemize}
					\item $x_E + 2x_I \leq 6$
					\item $2x_E + x_I \leq 8$
					\item $x_I - x_E \leq 1$
					\item $x_I \leq 2$
					\item $x_E, x_I \geq 0 $
				\end{itemize} 
			\begin{enumerate}
				\item Disegnamo lo \textbf{spazio delle soluzioni} di $x_E$ e $x_I$ che soddisfano tutti i vincoli.
				\begin{enumerate}
					\item Consideriamo nel primo vincolo solamente l'uguaglianza $x_E + 2x_I = b$
					\item Inseriamo i punti sul piano in base alle soluzioni del vincolo per una delle 2 variabili fissate. 
					\begin{enumerate}
						\item $x_E = 0 \rightarrow$ $x_I = 3$	
						\item $x_I = 0 \rightarrow$ $x_E = 6$			
					\end{enumerate}	
					\item Traccia la retta tra i 2 punti.
					\item Ripeti per ogni vincolo.
				\end{enumerate}
				\img{}{0.2}{img1.jpg}
				
				\item Ora cerchiamo di evidenzialre la \textbf{regione ammissibile}
					\begin{enumerate}
						\item Rappresentiamo i vettori \textbf{gradienti} per ogni vincolo, ciascuno rappresentato dai coefficienti del vincolo. $x_E + 2x_I \leq 6 \rightarrow \binom{1}{2}$
						\item Nel caso di vincoli con il $\geq$ il \textbf{gradiente} indica la parte dei punti soddisfatta dal vincolo, altrimenti l'opposto.
						\item L'intersezione tra tutti gli spazi da la regione ammissibile
						\img{}{0.2}{img2.jpg}
					\end{enumerate}
					
				\item Ora troviamo il punto che da la \textbf{soluzione ottima}.
				\begin{enumerate}
					\item Si genera un punto generico nella regione ammissibile.
					\item Sul punto si disegna il gradiente della funzione obiettivo.
					\item Poi la retta ortogonale al gradiente (\textbf{retta di Isocasto}) in cui tutti i punti hanno lo stesso costo.
					\item Il gradiente indica la direzione e il verso da seguire per aumentare il valore della funzione, quindi, dovendo massimizzare la funzione, spostiamo la retta in quella direzione fino a che la retta rimane nella regione ammissibile.
					\item Per verificarlo, dobbiamo essere sicuri che il \textbf{gradiente} della funzione obiettivo sia nel cono tra il gradiente del primo vincolo incontrato e del secondo.
					\img{}{0.25}{img3.jpg}
				\end{enumerate}
			\end{enumerate}
			
		\subsection{Tecnica di programmazione matematica}
			Formalizziamo ora un problema come \textbf{problema di programmazione matematica}:\\
			Problema ($f, X$) con 
			\begin{itemize}
				\item $f: R^n \rightarrow R$ funzione obiettivo.
				\item $X \subseteq R^n$ regione ammissibile
			\end{itemize}
			Quindi un problema per noi diventa $min f(\underline{x}) \quad \underline{x} \in X$.\\
			Con $\underline{x}$ definita: $\underline{x} \in X \iff \vx$ $soddisfa
			\begin{cases} 
				g_1(\underline{x}) \leq 0\\
				\dots \\
				g_m(\underline{x}) \leq 0 \\
				h_1(\underline{x}) = 0 \\
				\dots \\
				h_k(\underline{x}) = 0
			\end{cases}$
			
			\subsubsection{Convessità}
				Dati $\underline{x}$ e $\underline{y} \in R^n$ e lo scalare $\lambda \in [0,1]$, un vettore $\underline{z} \in R^n$ è una combinazione convessa di $\underline{x}$ e $\underline{y}$ se: $$\underline{z} = \lambda \underline{x} + (1- \lambda) \underline{y}$$
				
				\paragraph{Insieme convesso} Un insieme $S \subseteq R^n$ è convesso se ogni combinazione convessa di una qualunque coppia $\underline{x}, \underline{y} \in S$ appartiene ad $S$ stesso. L'intersezione di un qualunque numero di insiemi convessi è un insieme convesso.
			
				\paragraph{Funzione convessa} Una funzione $f: X \rightarrow R$ definita su di un insieme convesso $X \subseteq R^n$ si dice \textbf{convessa} se $\forall \underline{x}, \underline{y} \in X$ e $\forall \lambda \in [0,1]$ si ha che $$f(\underline{z}) \leq \lambda f( \underline{x} ) + (1-\lambda)f(\underline{y})\text{ con }\underline{z} = \lambda \underline{x} +(1-\lambda)\underline{y}$$
				\img{}{0.335}{img5.png}
				
				\paragraph{Funzione concava} La funzione $g$ è concava (sull'insieme convesso $X \subseteq R^n$) se $-g$ è convessa in $X$: $$g(\underline{z}) = g(\lambda \underline{x} + (1-\lambda) \underline{y}) \geq \lambda g(\underline{x}) + (1 - \lambda) g(\underline{y}) \quad \forall \underline{x}, \underline{y} \in X$$
				
				\paragraph{Minimo locale} $\underline{x} \in X$ è un minimo locale se esiste un intorno $N(\underline{x}) \subseteq X$ tale che $f(\underline{z}) \geq f(\underline{x})$ per ogni $\underline{z} \in N(\underline{x})$. $$N(\underline{x}) = \{\underline{z} : \underline{z} \in X \text{ e } ||\underline{x} - \underline{z} || \leq \epsilon \}$$	
				
				\paragraph{Teorema} Dato un problema di ottimizzazione convessa ($X,f$) ogni \textbf{minimo locale} è anche \textbf{minimo globale.}
					\img{}{0.3}{img6.png}
					
				\paragraph{Dimostrazione} La tesi da dimostrare è: $\forall \underline{y} \in X$ risulta $f(\underline{y}) \geq f(\underline{x})$.
					\begin{itemize}
						\item Il teorema vale se $\underline{y} \equiv \underline{z} \in N(\underline{x})$.
						\item Metto in relazione $\underline{x}, \underline{z}, \underline{y}$ e i corrispondenti valori della f.o.:
					\end{itemize}
					
					\begin{enumerate}
						\item Per trovare un controesempio, basta prendere un vettore $\underline{y}$ non appartenente all'interno di $N(\underline{x})$, che sia minore di $\underline{x}$ \textit{(minimo locale)}. $$\underline{y} \notin N(x) \quad f(\underline{x}) \leq f(\underline{y})$$
						\item Prendiamo qundi uno $\underline{z}$ che sia combinazione convessa di $\underline{x}$ e $\underline{y}$ e che appartenga all'interno. $$f(\underline{z}) = f(\lambda \underline{x} + (1- \lambda) \underline{y}) \leq \lambda f(\underline{x}) + (1-\lambda) f(\underline{y})$$
						\item $f(\underline{x}) \leq f(\underline{z})$ perchè $\underline{x}$ è minimo locale.
						\item
							\begin{itemize}
								\item $\fx \leq \lambda \fx + (1-\lambda) \fy$
								\item $\fx - \lambda \fx \leq (1-\lambda)$
								\item $(1-\lambda) \fx \leq (1-\lambda) \fy$ ma, essendo $\vx$ e $\vy$ diversi, $\lambda \ne 1$ e $\lambda \ne 0$, quindi:
								\item $\fx \leq \fy \quad \Box$
							\end{itemize}
					\end{enumerate}
				Noi ci interesseremo al mondo della \textbf{programmazione lineare}. $(f,X)$ è detto problema di programmazione lineare se e solo se la funzione obiettivo $f$ e tutte le funzioni che definiscono la regione ammissibile $X$ ($g_1(\vx), \dots, g_m(\vx), h_1(\vx), \dots, h_k(\vx)$ sono \textbf{lineari}, ossia sono concave e convesse contemporaneamente. 				Di conseguenza $X$ è convesso perchè intersezioni di insiemi (disuguaglianze) convessi.\\
				Perchè $X$ è definito da un insieme di disuguaglianze, il cui sistema genera una intersezione convessa, dimostriamo che un insieme definito in questo modo è a sua volta convesso.
				\paragraph{Dimostrazione}
					$X = \{ \vx \in R^n : \fx \leq 0 \} \quad$ $f$ convessa
					\begin{enumerate}
						\item Consideriamo 2 punti $\vx$ e $\vy$ tali che: $\fx \leq 0$ e $\fy \leq 0$
						\item Preso $\vz$ come \textbf{combinazione convessa} di $\vx$ e $\vy$ $(\vz = \lambda \vx + (1-\lambda)\vy)$ vale che $\fz \leq 0$: $$\fz \leq \underbrace{\underbrace{\lambda}_{\geq 0} \underbrace{\fx}_{\leq 0}}_{\leq 0} + \underbrace{\underbrace{(1-\lambda)}_{\geq 0} \underbrace{\fy}_{\leq 0}}_{\leq 0} \rightarrow \text{poichè $f$ è funzione convessa}$$ 
					\end{enumerate}
					Quindi ogni combinazione convessa di 2 punti che soddisfano la disuguaglianza, soddisfa a sua volta la disuguaglianza $\rightarrow$ $X$ insieme convesso. $\quad \Box$\\\\
				Ovviamente la programmazione lineare è solo un caso specifico di quella convessa.
		\subsection{Geometria della programmazione lineare}
			\paragraph{Iperpiano} \textit{(di supporto di un vincolo)} $\{\vx \in R^n: \underline{\alpha}^T \vx = \alpha_0 \}$
			\paragraph{Semispazio} $\{ \vx \in R^n: \va^T \vx \leq \alpha_0 \}$\\\\
			Con $\va$ vettore dei coefficienti, $\vx$ vettore delle variabili e $\alpha_0$ valore reale $\rightarrow$ $\va^T \vx$ prodotto scalare tra $\va$ e $\vx$ ($\va^T$ trasposta di $\va$).\\\\
			Iperpiano e semispazio \textbf{insiemi convessi} $\rightarrow$ la loro intersezione genera un insieme convesso.
			\paragraph{Poliedro} intersezione di un numero finito di \textbf{iperpiani} e \textbf{semispazi}.
			\paragraph{Politopo} poliedro $P$ limitato, ossia: $\exists M > 0: || \vx || \leq M \quad \forall	\vx \in P$
			\paragraph{Vertice} punto $x$ di un poliedro $P$ che non può essere espresso come \textbf{combinazione convessa \underline{stretta}} $(\lambda \ne 0,1)$ di altri 2 punti del poliedro $$\nexists \vy, \vz \in P, \vy \ne \vz, \lambda \in (0,1): \vx = \lambda \vy + (1-\lambda) \vz$$
			Ogni politopo ha un \textbf{numero finito di vertici}.
			
			\subsubsection{Forma matriciale del modello}
				\paragraph{Obiettivo} $max$ $3x_1 + 2x_2$
				\paragraph{Vincoli}
					\begin{itemize}
						\item $8x_1 + 4x_2 \leq 64$
						\item $4x_1 + 6x_2 \leq 54$
						\item $x_1 + x_2 \leq 10$
						\item $x_{1,2} \geq 0$
					\end{itemize}
				\paragraph{Vettore dei coefficienti dei costi} $\underline{c}^T = (3,2) $ vettore che contiene i coeeficienti della funzione obiettivo
				\paragraph{Vettore dei termini noti} 
					$ \underline{b} = \left (\begin{array}{l}
						64\\
						54\\
						10
					\end{array}\right)$ 
					vettore che contiene i termini noti posti dopo le disuguaglianze nei vincoli.
				\paragraph{Matrice dei coefficienti tecnologici} $ A = \left (\begin{array}{ll}
						8 & 4\\
						4 & 6\\
						1 & 1
					\end{array}\right)$ 
					matrice che contiene i coefficienti moltiplicativi delle variabili nei vincoli.
					
					Con $A_i$ indichiamo la colonna \textit{i-esima} di $A$, con $\va^T_i$ indichiamo la riga \textit{i-esima} di $A$.
					
				\paragraph{Vettore delle variabili} $ \vx = \left (\begin{array}{l}
						x_1\\
						x_2
					\end{array}\right)$ 
					vettore contenente tutte le variabili del modello.\\\\
				Quindi il problema è riformulabile in questi termini:
				\paragraph{Funzione obiettivo} $max$ $\underline{c}^T \vx$ 
				\paragraph{Vincoli} 
				\begin{itemize}
					\item $A \vx \leq \underline{b}$
					\item $\vx \geq 0$		
				\end{itemize}
				$\underline{P} = \{ \vx \in R^n | A |vx \leq b \land \vx \geq 0 \} \rightarrow$ insieme dei punti che soddisfano i vincoli, ossia la regione ammissibile (politopo o poliedro).\\\\
				Inoltre, in funzione della regione ammissibile $\underline{P}$, distinguiamo 3 tipi di problemi: 
				\begin{itemize}
					\item \textbf{Soluzione ottima finita} la cui regione ammissibile è un \textbf{politopo} (ossia una regione di spazio limitata), in cui quindi il numero di souzioni ottime è finito $(\underline{P} \ne \varnothing)$
					\item \textbf{Problema illimitato} la cui regione ammissibile è un \textbf{poliedro} (ossia una regione di spazio illimitata)
					\item \textbf{Problema inammissibile} la cui regione ammissibile è vuota, non ammettendo quindi soluzioni $(\underline{P} = \varnothing)$.
				\end{itemize}
				Nel primo caso, si possono verificare 3 diverse situazioni:
				\begin{itemize}
					\item \textbf{Soluzione unica} $\rightarrow$ esiste un unico punto \textit{(vertice)} che è soluzione ottima.
					\item \textbf{Ottimo multiplo} $\rightarrow$ l'insieme delle soluzioni ottime non è finito, poichè corrisponde una faccia del politopo. 
					\item \textbf{Soluzione degenere} $\rightarrow$ la soluzione  del problema è un vertice definito da più di 2 iperpiani (per definire un vertice è necessaria l'intersezione di soli 2 iperpiani) quindi quel punto nasconde più vertici (a causa dell'intersezione di tutte le coppie di iperpiani possibili).
				\end{itemize}
				
			\subsubsection{Teorema di Minkowski-Weil}
				Ogni punto di un politopo si può ottenere come combinazione convessa dei suoi vertici. 
				\paragraph{Teorema} Se $\vp = \{ \vx \in R^n | A \vx \leq b \land \vx \geq 0 \}$ è un politopo allora esiste almeno un vertice di $\vp$ ottimo per il problema $min$ $\{ c^T \vx | \vx \in \vp \}$
				
				\paragraph{Dimostrazione}
					\begin{enumerate}
						\item Siano $\vy_1, \dots, \vy_k$ i vertici del politopo $\vp$ e sia $z^* = \{ min$ $\underline{c}^T \vy_j$ con $j=1, \dots, k \} \quad$ \textit{($z^*$ non è nient'altro che il valore minimo della funzione obiettivo calcolata in tutti i vertici $\vy_j$.)}
						\item Sia $\vx \in \vp$ un punto qualsiasi del politopo, non vertice. Per Minkwoski-Weil questo punto è generabile come combinazione convessa di $k$ vertici di $\vp$.
						\item $\exists \underline{\lambda} \in [0,1]^n$ $|$ $\vx = \sum_{j=1}^k \lambda_j \vy_j$ con $\sum_{j=1}^k \lambda_j = 1$
						\item Quindi la funzione obiettivo per il punto generico $\vx$ è esprimibile
						$$\underline{c}^T \vx = \underline{c}^T (\sum_{j=1}^k \lambda_j \vy_j) = \sum_{j=1}^k \lambda_j \underbrace{\underline{c}^T \vy_j}_{\substack{\geq z^* \\ \text{perchè} \\ \text{sono} \\ \text{i valori} \\ \text{di $f$}\\ \text{per i} \\ \text{vertici}}} \geq \sum_{j=1}^k \lambda_j z^* = z^* \underbrace{\sum_{j=1}^j \lambda_j}_{=1} = z^* \quad \quad \forall \vx \in \vp \quad \underline{c}^T \vx \geq z^* \quad \Box$$								\end{enumerate}
			
			\subsubsection{Come leghiamo vertici e matrici?}
				La \textbf{forma standard} è la forma usata dagli algoritmi: 
				\paragraph{Funzione obiettivo} $min$ $c^Tx$
				\paragraph{Vincoli} $A \vx = \underline{b}, \quad x \geq 0$\\\\
				Ma come passiamo da una forma generica a quella standard?
				
				\begin{enumerate}
					\item Per trasformare un vincolo generico caratterizzato da una disuguaglianza a uno con uguaglianza usiamo una \textbf{variabile aggiuntiva} \textit{(detta di scarto, slack o di surplus)}:
					\begin{itemize}
						\item $\va^T \vx \leq b \rightarrow \va^T \vx + s = b$ con $s \geq 0$ 
							\begin{itemize}
								\item Per tutti i punti $\vx$ per cui $\va^T \vx = b$ (vincolo attivo), $s = 0$.
								\item Per tutti i punti $\vx$ per cui $\va^T \vx < b$ (vincolo NON attivo), $s>0$.
							\end{itemize}
						\item$\va^T \vx \geq b \rightarrow \va^T \vx -s = b$ con $s \geq 0$
					\end{itemize}
					\item Se abbiamo una variabile $x_j \leq 0$ che quindi non soddisfa il vincolo di non negatività, usiamo una variabile differente:
					\begin{itemize}
						\item $x_j \leq 0 \rightarrow \overline{x_j} = -x_j \quad (-x_j \geq 0)$
						\item $x_j$ libera $\rightarrow x_j^+ - x_j^- = x_j \quad (x_j^+, x_j^- \geq 0)$
						\begin{itemize}
							\item $x_j$ non ha vincoli sul segno \textit{(può essere sia positiva che negativa).}
						\end{itemize}
					\end{itemize}
					\item Nel casi di funzioni obiettivo con massimo, la possiamo trasformare in funzione di minimo: $max$ $\vc^T \vx = - min$ $-\vc^T \vx$
				\end{enumerate}
				Ora vediamo perchè è utile usare la \textbf{forma standard}:					
				
\end{document}

